% This is LLNCS.DOC the documentation file of
% the LaTeX2e class from Springer-Verlag
% for Lecture Notes in Computer Science, version 2.4
%%\documentclass{llncs}

\documentclass{llncs}
\usepackage{llncsdoc}
\usepackage[english]{babel}

\begin{document}
\markboth{Gestenerkennung in ROS mit Pepper
}{Gestenerkennung in ROS mit Pepper}
\thispagestyle{empty}
\begin{flushleft}
\LARGE\bfseries ISE Master Seminararbeit\\
in Robotik\\[2cm]
\end{flushleft}
\rule{\textwidth}{1pt}
\vspace{2pt}
\begin{flushright}
\Huge
\begin{tabular}{@{}l}
Gestenerkennung\\
in ROS\\
mit Pepper\\[6pt]
\end{tabular}
\end{flushright}
\rule{\textwidth}{1pt}
\vfill
%
\newpage
\tableofcontents
\newpage
%

\begin{abstract}
Die Gestenerkennung ist ein wichtiges Teilgebiet der Mensch-Maschine-Interaktion. Sie soll den Umgang und die Kommunikation von Menschen mit Maschinen verbessern und kommt bereits im Alltag zur Anwendung, beispielsweise bei der Nutzung von Mobiltelefonen oder Spielkonsolen. Im Rahmen dieser Arbeit werden verschiedene Verfahren zur Gestenerkennung vorgestellt und evaluiert. Darüber hinaus wird untersucht, wie gut diese Verfahren auf dem humanoiden Roboter \glqq Pepper\grqq\hspace{1pt} umzusetzen sind.

%Es werden zun\"achst drei Verfahren zur Gestenerkennung vorgestellt. Anschließen%d wird evaluiert, wie gut 



%% %%Abstract 10-15zeilen: mmi gesten als wichtige modalität gewinnbringend nutzen
%%  In Hinblick auf die Nutzung von Mensch-Maschine-Interaktion ist die Gestenerkennung eine wichtige Modalit\"at. Sie vereinfacht den Umgang und die Kommunikation von den Menschen mit einer Maschine.  
\end{abstract}
%
\section{Introduction}
In der heutigen Zeit, wo Arbeitspl\"atze, die unter anderem auch die interaktion mit Menschen beinhalten, durch humanoide Roboter ersetzt werden, gewinnt die Mensch-Maschine-Interaktion mehr an Bedeutung. Die Verfahren m\"ussen sicher und fehlerfrei funktionieren, um den Betrieb am laufen zu halten. Damit die Bedienung der Roboter vereinfacht wird, kann Gestensteuerung gewinnbringend genutzt werden.
%
\section{Gestensteuerung}
Hier kommt die Begründung hin, warum gestensteuerung benutzt wird
Anhand von Gesten k\"onnen Roboter gesteuert werden. Dadurch kann die interaktion im Alltag erleichtert werden. 
% warum gestensteuerung
%% warum gestenerkennung 
\section{Pepper}
 was kann pepper, was braucht man (eingabedatenkan\"ale ) hautfarbe, 2d, 3d andere sensoren?
\begin{itemize}
\item ca 1,20m gro\ss{} und 30kg schwer
\item vier Mikrofone, zwei RGB-Kameras und ein 3D-Sensor im Kopf
\item Gyrosensor im Torso
\item Ber\"uhrungssensoren in Kopf und H\"anden
\item Fu\ss{}bereich enth\"alt 2 Sonarsensoren, 6 Laserscanner, 3 Sto\ss{}f\"angersensoren und einen weiteren Gyrosensor.
\end{itemize}
\section{Verfahren zu Gestenerkennung}
In diesen Kapitel folgen drei Verfahren zur Gestenerkennung.
%% → verfahren, wenn mehere dann evaluieren, was passt zu szenario
%%	zielszenario: nur formen, oder bewegte gesten \"uber zeit.

\subsection{Gestenerkennung durch KOMSM}
Hier wird das KOMSM Verfahren vorgestellt.\\
Verfahren bentutz keine Bewegten Gesten, sondern einzelne Positionen von Händen, die analysiert werden. 
I Introduction
\begin{itemize}
\item Kinect sensor als Revolution
\item Vorteil: Geringe Rechenkosten und geringe Anzahl an Parametern, die gesetzt werden müssen
\item Nachteil: 2 verschiedene Tiefenbilder n\"otig
\item MSM erweitert durch mehrere nichtlineare Methoden zu ``Kernel Orthogonal Mutual Subspace Method''(KOMSM)
\item für jede Klasse aus einer Menge von Tiefenbildern wird eine Referenz erzeugt, die ``Volume Subspace'' genannt
\item Ähnlichkeit zwischen input Subspace und Volume Subspace wird mit kanonischen Winkeln gemessen (Bild)
\subitem invariant zu Beleuchtungszustand und Objektfarbe
\subitem invariant zu Translation und Skalierung des Objekts durch simple Normalisierung von Gr\"oße und Tiefe
\item Gesten werden in Befehle umgewandelt
\end{itemize}
II Vorgeschlagenes Gestenerkennungssystem
\begin{itemize}
\item A: Volume subspace
\subitem Wenn die Nummer der kanonischen betrachteten Winkel gr\"o\ss{}er als 3 ist, ist die 3D-Information des Objektes bereits im Volume Subspace enthalten
\subitem Vorteil: Beleuchtung egal, da Tiefenbilder nicht drauf angewiesen sind
\item B: Recognition Based on KOMSM
\subitem Measurement von 2 Subspaces: 
\subsubitem subspaces werden durch Kernel PCA generiert
\subsubitem die Kanonical Winkel von 2 nichtlinearen Subspaces werden benutzt,m um die Gleichheit der Subspaces zu quantifizieren.
\subsubitem Erkl\"arung der canonical Winkel(Formelkram)
\subitem Orthogonalität von nichtlinearen Subspaces: 
\subsubitem Die nichtlinearen Subspaces werden orthogonalisiert, bevor die Kanonischen Winkel angewendet werden.
\subitem Definition of Similarity
\subsubitem (Formel)
\subitem Reduktion von Rechenzeit:
\subsubitem Nachteil: Rechenzeit wächst im Verhältnis zur Anzahl der Lernmuster
\subsubitem Um die Komplexität der Rechnung zu reduzieren werden die Lernmuster in k Kluster gepackt, indem k-means benutzt werden und dann die Kernorthogonalisierung wird vom Zentroid der enthaltenen K-Klusters aus berechnet
\subsubitem Wenn k k kleiner als die Nummer von Lernpattern ist, ist die Rechenzeit deutlich reduziert. 
\item C: Handdetection
\subitem In diesem Verfahren werden die Funktionalitäten von ROS und OpenNI benutzt, um das Skelett einer Person zu tracken
\subitem Position der Hand wird als Pointcloud extrahiert
\subitem ROS Paket ``MIT Kinect Demos'', modifiziert, damit neben 2 Pointclouds auch das abgeschnittene RGB- und Tiefenbild beider Hände zu erhalten
\subitem Tiefenbilder werden auf eine Größe von 16x16 Pixel komprimiert, die Tiefenwerte werden normalisiert
\subitem Das RGB-Bild wird für die Visualisierung benutzt (Kann ich vlt weg lassen)
\item D: Handgesten Definition:
\subitem Es wurde ein Set von Gesten entworfen, um den Roboter damit zu kontrollieren
\subitem Jede Geste hat eine Action-Nummer, wenn der Roboter eine erkennt, wird der entsprechende Befehl an den Roboter gesendet.
\subitem Wenn man die Hände für die Geste vertauscht, sollte das System den Befehl trotzdem erkennen, muss also doppelt so viele Gesten kennen. Bei so einer Anzahl wird es sehr rechenintensiv und fehleranfällig
\subsubitem Nur eine kleine Anzahl an Gesten zur Verfügung stellen(Hier 9) plus eine extra Klasse, um Nebeneffekte aus dem Handsegmentierungssystem zu verhindern
\subitem Damit beide Hände gleich aussehen, wird das Bild der linken Hand horizontal gedreht 
\subsubitem Somit können n quadrat verschiedene Gesten differenziert werden 
\subitem Nicht alle Kombinationen haben eine Bedeutung
\subitem Alle gespeicherten Gesten können in eine Matrix gepackt werden 
\item E: Roboter-Kontrolle:
\subitem (Aufbau des Roboters ist hier unwichtig)
\subitem Aufbau von ROS mit nodes
\subitem Node, um RGB- und Tiefenbild aufzunehmen, einer um das humane Skelett in Tiefenbild zu erkennen, ...
\subitem ROS läuft auf einen Desktop PC, der die Befehle über WLAN an den Roboter schickt
\end{itemize}
III Ablauf der vorgeschlagenen Methode
\begin{itemize}
\item A: Lernphase
\subitem 1. Referenziere Bilder von n unterschiedlichen Gesten der rechten Hand werden aufgenommen
\subitem 2. Die Größe und die Tiefenwerte der Referenz-Bilder werden normalisiert
\subitem 3. Es wird ein Volume Subspace für jeden der n Gesten generiert und gespeichert, indem das KOMSM Netzwerk benutzt wird. 
\item B: Erkennungsphase
\subitem 1. Bilder der linken und rechten Hand werden aufgenommen
\subitem 2. Beide Bilder werden in der Größe normalisiert
\subitem 3. Aufnahme der linken Hand wird horizontal gedreht
\subitem 4. Die Tiefenwerte beider Bilder werden normalisiert
\subitem 5. Volume Subspaces für rechte und linke Hand werden mittels KOMSM erstellt. Kanonische Winkel werden berechnet. 
\subitem 6. Matrix für die Ähnlichkeit der both-hand Gesten wird aufgestellt.
\subitem 7. Es werden die beidhändigen Gesten gefunden. 
\end{itemize}
IV Experimente
\subsection{Gestenerkennung mit Tiefensensor}
Hier wird das Gestenerkennung mit Tiefensensor Verfahren vorgestellt.
\subsection{Gestenerkennung durch neuronale Netzwerke}
Hier wird die Gestenerkennung durch neuronale Netzwerke vorgestellt.
\section{Fazit}
Verfahren xy eignet sich in dieser Situation am meisten.
\newpage
\begin{thebibliography}{[MT1]}
\bibitem[Kin]{Kinect}
Gu, Y., Do, H., Ou, Y., Sheng, W.:
Human Gesture Recognition through a Kinect Sensor
2014 IEEE-RAS International Conference on Humanoid Robots (2014)
\bibitem[Nea]{Neuronal}
Barros P, Parisi G., Jirak D., Wermter S.:
Real-time Gesture Recognition Using a Humanoid Robot with a Deep Neural Architecture 2014 IEEE-RAS International Conference on Humanoid Robots (2014)
\bibitem[Pep0]{Pepper}
Lafaye J.,Gouaillier D., Wieber P.:
Linear model predictive control of the locomotion of Pepper, a humanoid robot with omnidirectional wheels 2014 IEEE-RAS International Conference on Humanoid Robots
\bibitem[Pep1]{Pepper2}
Tanaka, F.,Isshiki K., Takahashi F.,Uekusa M., Sei R., Hayashi K.:
Pepper learns together with children: Development of an educational application 2015 IEEE-RAS 15th International Conference on Humanoid Robots (Humanoids)
%http://ieeexplore.ieee.org/document/7363546/#
\end{thebibliography}

\end{document}
