% This is LLNCS.DOC the documentation file of
% the LaTeX2e class from Springer-Verlag
% for Lecture Notes in Computer Science, version 2.4
\documentclass{llncs}
\usepackage{llncsdoc}
%
\begin{document}
\markboth{\LaTeXe{} Class for Lecture Notes in Computer
Science}{\LaTeXe{} Class for Lecture Notes in Computer Science}
\thispagestyle{empty}
\begin{flushleft}
\LARGE\bfseries Instructions for Authors\\
Coding with \LaTeX\\[2cm]
\end{flushleft}
\rule{\textwidth}{1pt}
\vspace{2pt}
\begin{flushright}
\Huge
\begin{tabular}{@{}l}
Gesture Recognition\\
in ROS\\
with Pepper\\[6pt]
\end{tabular}
\end{flushright}
\rule{\textwidth}{1pt}
\vfill
\newpage
%
\section*{For further information please contact us:}
%
\begin{flushleft}
\begin{tabular}{l@{\quad}l@{\hspace{3mm}}l@{\qquad}l}
$\bullet$&\multicolumn{3}{@{}l}{\bfseries LNCS Editorial Office}\\[1mm]
&\multicolumn{3}{@{}l}{Springer-Verlag}\\
&\multicolumn{3}{@{}l}{Computer Science Editorial}\\
&\multicolumn{3}{@{}l}{Tiergartenstraße 17}\\
&\multicolumn{3}{@{}l}{69121 Heidelberg}\\
&\multicolumn{3}{@{}l}{Germany}\\[0.5mm]
 & Tel:       & +49-6221-487-8706\\
 & Fax:       & +49-6221-487-8588\\
 & e-mail:    & \tt lncs@springer.com    & for editorial questions\\
 &            & \tt texhelp@springer.de & for \TeX{} problems\\[2mm]
\noalign{\rule{\textwidth}{1pt}}
\noalign{\vskip2mm}
%
%{\tt svserv@vax.ntp.springer.de}\hfil first try the \verb|help|
%command.
%
$\bullet$&\multicolumn{3}{@{}l}{\bfseries We are also reachable through the world wide web:}\\[1mm]
         &\multicolumn{2}{@{}l}{\texttt{http://www.springer.com}}&Springer Global Website\\
         &\multicolumn{2}{@{}l}{\texttt{http://www.springer.com/lncs}}&LNCS home page\\
         &\multicolumn{2}{@{}l}{\texttt{http://www.springerlink.com}}&data repository\\
         &\multicolumn{2}{@{}l}{\texttt{ftp://ftp.springer.de}}&FTP server
\end{tabular}
\end{flushleft}


%
\newpage
\tableofcontents
\newpage
%
\section{Introduction}
%
Abstract 10-15zeilen: mmi gesten als wichtige modalität gewinnbringend nutzen
%%
Authors wishing to code their contribution
with \LaTeX{}, as well as those who have already coded with \LaTeX{},
will be provided with a document class that will give the text the
desired layout. Authors are requested to
adhere strictly to these instructions; {\em the class
file must not be changed}.

The text output area is automatically set within an area of
12.2\,cm horizontally  and 19.3\,cm vertically.

If you are already familiar with \LaTeX{}, then the
LLNCS class should not give you any major difficulties.
It will change the layout to the required LLNCS style
(it will for instance define the layout of \verb|\section|).
We had to invent some extra commands,
which are not provided by \LaTeX{} (e.g.\
\verb|\institute|, see also Sect.\,\ref{contbegin})
%
\section{Gestensteuerung}
%
The LLNCS class is an extension of the standard \LaTeX{} ``article''
document class. Therefore you may use all ``article'' commands for the
body of your contribution to prepare your manuscript.
LLNCS class is invoked by replacing ``article'' by ``llncs'' in the
first line of your document:
\begin{verbatim}
\documentclass{llncs}
%
\begin{document}
  <Your contribution>
\end{document}
\end{verbatim}
%
\section{Pepper}
Hier kommen zum Beispiel die technischen Daten von Pepper hin. 
-> was kann pepper, was braucht man (eingabedatenkanäle ) hautfarbe, 2d, 3d andere sensoren?
\section{Verfahren zu Gestenerkennung}
%
Hier folgen drei Verfahren zur Gestenerkennung.
 → verfahren, wenn mehere dann evaluiren, was passt zu szenario
	zielszenario: nur formen, oder bewegte gesten über zeit.
	warum gestenerkennung
\subsection{Gestenerkennung durch KOMSM}
KOMSM
\subsection{Gestenerkennung mit Tiefensensor}
aa
\subsection{Gestenerkennung durch neuronale Netzwerke}
lll
%
\section{Fazit}
Alles sehr toll.
\section{References}
\label{refer}
%
There are three reference systems available; only one, of course,
should be used for your contribution. With each system (by
number only, by letter-number or by author-year) a reference list
containing all citations in the
text, should be included at the end of your contribution placing the
\LaTeX{} environment \verb|thebibliography| there.
For an overall information on that environment
see the {\em \LaTeX{} User's Guide \& Reference
Manual\/} by Leslie Lamport, p.~71.

There is a special {\sc Bib}\TeX{} style for LLNCS that works along
with the class: \verb|splncs.bst|
-- call for it with a line \verb|\bibliographystyle{splncs}|.
If you plan to use another {\sc Bib}\TeX{} style you are customed to,
please specify the option \verb|[oribibl]| in the
\verb|documentclass| line, like:
\begin{verbatim}
\documentclass[oribibl]{llncs}
\end{verbatim}
This will retain the original \LaTeX{} code for the bibliographic
environment and the \verb|\cite| mechanism that many {\sc Bib}\TeX{}
applications rely on.
%
\subsection{References by Letter-Number or by Number Only}
%
References are cited in the text -- using the \verb|\cite|
command of \LaTeX{} -- by number or by letter-number in square
brackets, e.g.\ [1] or [E1, S2], [P1], according to your use of the
\verb|\bibitem| command in the \verb|thebibliography| environment. The
coding is as follows: if you choose your own label for the sources by
giving an optional argument to the \verb|\bibitem| command the citations
in the text are marked with the label you supplied. Otherwise a simple
numbering is done, which is preferred.
\begin{verbatim}
The results in this section are a refined version
of \cite{clar:eke}; the minimality result of Proposition~14
was the first of its kind.
\end{verbatim}
The above input produces the citation: ``\dots\ refined version of
[CE1]; the min\-i\-mality\dots''. Then the \verb|\bibitem| entry of
the \verb|thebibliography| environment should read:
\begin{verbatim}
\begin{thebibliography}{[MT1]}
.
.
\bibitem[CE1]{clar:eke}
Clarke, F., Ekeland, I.:
Nonlinear oscillations and boundary-value problems for
Hamiltonian systems.
Arch. Rat. Mech. Anal. 78, 315--333 (1982)
.
.
\end{thebibliography}
\end{verbatim}
The complete bibliography looks like this:
%
\begin{thebibliography}{[MT1]}
%
\bibitem[CE1]{clar:eke}
Clarke, F., Ekeland, I.:
Nonlinear oscillations and
boundary-value problems for Hamiltonian systems.
Arch. Rat. Mech. Anal. 78, 315--333 (1982)
%
\bibitem[CE2]{clar:eke:2}
Clarke, F., Ekeland, I.:
Solutions p\'{e}riodiques, du
p\'{e}riode donn\'{e}e, des \'{e}quations hamiltoniennes.
Note CRAS Paris 287, 1013--1015 (1978)
%
\bibitem[MT1]{mich:tar}
Michalek, R., Tarantello, G.:
Subharmonic solutions with prescribed minimal
period for nonautonomous Hamiltonian systems.
J. Diff. Eq. 72, 28--55 (1988)
%
\bibitem[Ta1]{tar}
Tarantello, G.:
Subharmonic solutions for Hamiltonian
systems via a $\bbbz_{p}$ pseudoindex theory.
Annali di Matematica Pura (to appear)
%
\bibitem[Ra1]{rab}
Rabinowitz, P.:
On subharmonic solutions of a Hamiltonian system.
Comm. Pure Appl. Math. 33, 609--633 (1980)
\end{thebibliography}
%
\subsubsection*{Number-Only System.}
%
For this preferred system do not use the optional argument
in the \verb|\bibitem| command: then, only numbers will
appear for the citations in the text (enclosed in square brackets)
as well as for the marks in your
bibliography (here the number is only end-punctuated without
square brackets).

Subsequent citation numbers in the text are collapsed to ranges.
Non-numeric and undefined labels are handled correctly but no sorting is
done.

E.g., \verb|\cite{n1,n3,n2,n3,n4,n5,foo,n1,n2,n3,?,n4,n5}| -- where
\verb|n|$x$ is the key of the $x^{\mathrm{th}}$ \verb|\bibitem|
command in sequence, \verb|foo| is the key of a \verb|\bibitem| with an
optional argument, and \verb|?| is an undefined reference -- gives
1,3,2-5,foo,1-3,?,4,5 as the citation reference.

\begin{verbatim}
\begin{thebibliography}{1}
\bibitem {clar:eke}
Clarke, F., Ekeland, I.:
Nonlinear oscillations and boundary-value problems for
Hamiltonian systems.
Arch. Rat. Mech. Anal. 78, 315--333 (1982)
\end{thebibliography}
\end{verbatim}
%
\subsection{Author-Year System}
%
References are cited in the text by name and year in parentheses
and should look as follows:
(Smith 1970, 1980), (Ekeland et al. 1985, Theorem 2), (Jones and Jaffe
1986; Farrow 1988, Chap.\,2). If the name is part of the sentence
only the year may appear in parentheses,
e.g.\ Ekeland et al. (1985, Sect.\,2.1)
The reference list should contain all citations occurring in the text,
ordered alphabetically by surname (with initials following). If there
are several works by the same author(s) the references should be listed
in the appropriate order indicated below:
\begin{alpherate}
\setlength{\hfuzz}{5pt}
\item
One author: list works chronologically;
\item
Author and same co-author(s): list works chronologically;
\item
Author and different co-authors: list works alphabetically
according to co-authors.
\end{alpherate}
If there are several works by the same author(s) and in the same year,
but which are cited separately, they should be distinguished by the use
of ``a'', ``b'' etc., e.g.\ (Smith 1982a), (Ekeland et al. 1982b).
%
\subsubsection*{How to Code Author-Year System.}
%
If you want to use this system you have to specify the option
\verb|[citeauthoryear]| in the \verb|documentclass|, like:
\begin{verbatim}
\documentclass[citeauthoryear]{llncs}
\end{verbatim}
Write your citations in the text explicitly except for the year, leaving
that up to \LaTeX{} with the \verb|\cite| command. Then give only the
appropriate year as the optional argument (i.e. the label in square
brackets) with the \verb|\bibitem| command(s).\\[2mm]
{\itshape Sample Input}
\begin{verbatim}
The results in this section are a refined version
of Clarke and Ekeland (\cite{clar:eke}); the minimality result of
Proposition~14 was the first of its kind.
\end{verbatim}
The above input produces the citation: ``\dots\ refined version of
Clarke and Ekeland (1982); the minimality\dots''. Then the
\verb|\bibitem| entry of \verb|clar:eke| in the \verb|thebibliography|
environment should read:
\begin{verbatim}
\begin{thebibliography}{}  % (do not forget {})
.
.
\bibitem[1982]{clar:eke}
Clarke, F., Ekeland, I.:
Nonlinear oscillations and boundary-value problems for
Hamiltonian systems.
Arch. Rat. Mech. Anal. 78, 315--333 (1982)
.
.
\end{thebibliography}
\end{verbatim}
{\itshape Sample Output}
\bibauthoryear
%
\end{document}
