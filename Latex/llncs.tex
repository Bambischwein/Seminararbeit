\documentclass{llncs}
\usepackage{llncsdoc}
\usepackage[english]{babel}
\begin{document}
\markboth{Gestenerkennung in ROS (mit Pepper)
}{Gestenerkennung in ROS mit Pepper}
\thispagestyle{empty}
\begin{flushleft}
\LARGE\bfseries ISE Master Seminararbeit\\
in Robotik\\[2cm]
\end{flushleft}
\rule{\textwidth}{1pt}
\vspace{2pt}
\begin{flushright}
\Huge
\begin{tabular}{@{}l}
Gestenerkennung\\
in ROS\\
(mit Pepper)\\[6pt]
\end{tabular}
\end{flushright}
\rule{\textwidth}{1pt}
\vfill
\newpage
\tableofcontents
\newpage
\begin{abstract}
Die Gestenerkennung ist ein wichtiges Teilgebiet der Mensch-Maschine-Interaktion. Sie soll den Umgang und die Kommunikation von Menschen mit Maschinen verbessern und kommt bereits im Alltag zur Anwendung, beispielsweise bei der Nutzung von Mobiltelefonen oder Spielkonsolen. Im Rahmen dieser Arbeit werden verschiedene Verfahren zur Gestenerkennung vorgestellt und evaluiert. Dar\"uber hinaus wird untersucht, wie gut diese Verfahren auf dem humanoiden Roboter \glqq Pepper\grqq\hspace{1pt} umzusetzen sind.

%% %%Abstract 10-15zeilen: mmi gesten als wichtige modalität gewinnbringend nutzen
%%  In Hinblick auf die Nutzung von Mensch-Maschine-Interaktion ist die Gestenerkennung eine wichtige Modalit\"at. Sie vereinfacht den Umgang und die Kommunikation von den Menschen mit einer Maschine.  
\end{abstract}

%\begin{abstract}
Die Gestenerkennung ist ein wichtiges Teilgebiet der Mensch-Maschine-Interaktion. Sie soll den Umgang und die Kommunikation von Menschen mit Maschinen verbessern und kommt bereits im Alltag zur Anwendung, beispielsweise bei der Nutzung von Mobiltelefonen oder Spielkonsolen. Im Rahmen dieser Arbeit werden verschiedene Verfahren zur Gestenerkennung vorgestellt und evaluiert. Dar\"uber hinaus wird untersucht, wie gut diese Verfahren auf dem humanoiden Roboter \glqq Pepper\grqq\hspace{1pt} umzusetzen sind.

%% %%Abstract 10-15zeilen: mmi gesten als wichtige modalität gewinnbringend nutzen
%%  In Hinblick auf die Nutzung von Mensch-Maschine-Interaktion ist die Gestenerkennung eine wichtige Modalit\"at. Sie vereinfacht den Umgang und die Kommunikation von den Menschen mit einer Maschine.  
\end{abstract}
% Newpage ist kacke
%\section{Introduction}
In der heutigen Zeit, wo Arbeitspl\"atze, die unter anderem auch die interaktion mit Menschen beinhalten, durch humanoide Roboter ersetzt werden, gewinnt die Mensch-Maschine-Interaktion mehr an Bedeutung. Die Verfahren m\"ussen sicher und fehlerfrei funktionieren, um den Betrieb am laufen zu halten. Damit die Bedienung der Roboter vereinfacht wird, kann Gestensteuerung gewinnbringend genutzt werden.\\
- alte verfahren mit zb Handschuh zu erkennung aufführen und dann sagen, dass dsas zu aufwendig ist und ein verfahren ohne zusatzmaterial gesucht wird.  Literaturverfahren

\section{Introduction}
In der heutigen Zeit, wo Arbeitspl\"atze, die unter anderem auch die interaktion mit Menschen beinhalten, durch humanoide Roboter ersetzt werden, gewinnt die Mensch-Maschine-Interaktion mehr an Bedeutung. Die Verfahren m\"ussen sicher und fehlerfrei funktionieren, um den Betrieb am laufen zu halten. Damit die Bedienung der Roboter vereinfacht wird, kann Gestensteuerung gewinnbringend genutzt werden.\\
%- alte verfahren mit zb Handschuh zu erkennung aufführen und dann sagen, dass dsas zu aufwendig ist und ein verfahren ohne zusatzmaterial gesucht wird.  Literaturverfahren

\section{Pepper}
 was kann pepper, was braucht man (eingabedatenkan\"ale ) hautfarbe, 2d, 3d andere sensoren?

\section{Gestensteuerung}
Hier kommt die Begr\"undung hin, warum gestensteuerung benutzt wird
Anhand von Gesten k\"onnen Roboter gesteuert werden. Dadurch kann die interaktion im Alltag erleichtert werden. \\
Eine Geste ist eine Bewegung des K\"orpers, die Informationen ent\"alt.
% warum gestensteuerung
%% warum gestenerkennung 
\begin{itemize}
\item ca 1,20m gro\ss{} und 30kg schwer
\item vier Mikrofone, zwei RGB-Kameras und ein 3D-Sensor im Kopf
\item Gyrosensor im Torso
\item Ber\"uhrungssensoren in Kopf und H\"anden
\item Fu\ss{}bereich enth\"alt 2 Sonarsensoren, 6 Laserscanner, 3 Sto\ss{}f\"angersensoren und einen weiteren Gyrosensor.
\end{itemize}
\section{Verfahren zu Gestenerkennung}
Einleitung zu den Verfahren der Gestensteuerung.

\include{Verfahren/KOMSM}
\subsection{Gestenerkennung mit Tiefensensor (Kinect)}
Hier wird das Gestenerkennung mit Tiefensensor Verfahren vorgestellt. 
%F\"ur die Durchf\"uhrung muss die Testperson kurze \"Armel tragen und im Hintergrund d\"urfen sich keine hautfarbenen Gegenst\"ande befinden.
\begin{itemize}
\item HMM, um die dynamik der Gesten zu erkennen und zu modelieren
\item 3 Schichten: Detection, Tracking und Recognition
\subitem Detection und Tracking werden durch Kinect Middleware umgesetzt
\subitem Diese Arbeit ist auf Klassifikation fokussiert
\end{itemize}
III Implementation
\begin{itemize}
\item Für dieses Paper wurden 5 Gesten gespeichert(Kommen, Gehen, Winken, Aufstehen, Hinsetzen)
\subitem Jede Geste wird durch ein HMM encoded
\item Alle Gesten werden mit dem linken Arm ausgef\"uhrt %(unflexibel) 
\item 4 joint angles: Ellbogen (yaw and roll), Schulter (yaw and pitch)
\end{itemize}
A: Training phase
\begin{itemize}
\item Jedes Model wird mit 15 Datensets (20Hz) trainiert
\item Eine Regel-basierte Methode wird für die Daten-Segmentierung hinzugef\"ugt
\item Es wird eine Startpose definiert und jedes Datenset besteht aus 30 Datenpunkten nach dem Startpunkt
\item Jede Geste hat die gleiche Dauer (eineinhalb Sek)
\item ``K-means clustering'' wird benutzt, um die Vektoren in sichtbare Symbole f\"ur HMM zu konvertieren
\item Die Zentroiden der k-means werden f\"ur die Testdaten gesichert
\item Um die Rechenkomplexit\"at auszubalancieren werden die HMM-Parameter gesetzt: Number of States: 30, Number of distinct observation symbols: 6
\end{itemize}
B: Recognition Phase
\begin{itemize}
\item Es werden die Trainings-HMMs mit 2 verschiedenen Testobjekten getestet 
\subitem Das Testobjekt, das teilgenommen hat, und eins, das nicht teilgenommen hat
\item F\"ur real-time processing wird ROS eingesetzt
\subitem Jede Funktion wird als Node geschrieben, die simultan laufen (Hier werden 3 Nodes genutzt)
\subitem Image extraction: Liest die Daten aus der Kinect
\subitem Feature extraction: kreiert das skelett model basierend auf den Bildinformationen
\subitem  real-time recognition: Buffert die joint angles informationen vom Feature extracion node und kalkuliert danach die Wahrscheinlichkeit der Beobachtungssequenz, die f\"ur jedes Model gegeben ist. Dsa model mit der gr\"ossten Wahrscheinlichkeit bestimmt den Typen des HMM
\item Um St\"orpunkte loszuwerden und die Fehlalarm-rate zu reduzieren wird zuerst die Varianz des Inputs benutzt, um zu entscheiden, ob es eine Geste ist oder nicht
\item Danach wird ein Schwellwert für jedes HMM gesetzt
\item Wenn die Wahrscheinlichkeit kleiner als der Schwellwert ist wird es als St\"orung behandelt
\end{itemize}
IV: Experimental Results
A: Recognition Results
\begin{itemize}
\item Bei unterschiedlicher Geschwindigkeit: langsam und normal wurden 4 von 5 erkannt, bei zu schneller Bewegung nur 1 von 5, weil Keypoints nicht schnell genug aufgenomen werden k\"onnen
\item Solange es einmal detektiert wird, bleibt es als Erkannt klassifiziert
\item Bearbeitungszeit ist sehr kurz verglichen mit der Datenaufnahme, also gehen dadurch keine Daten verloren
\item Es kann die gleiche Geste ohne Pause wiederholt werden und sie wird immer erkannt
\item die benutzen Gesten sind sehr unterschiedlich (wie sieht es dann mit \"ahnlichen Gesten aus?)
\item Solange der Algorithmus das Skelett anhand der Bilder von der Kamera bilden kann funktioniert die Erkennung
\end{itemize}
V: Conclusion and future works
\begin{itemize}
\item Einfaches Training: Geste muss nur einmal aufgenommen werden, um erkannt zu werden
\item System kann von unterschiedlichen Personen trainiert und benutzt werden
\item Orientierung der Gesten m\"ussen nicht mit den  Orientierungen der aufgenommenen Gesten  \"ubereinstimmen
\item Geschwindigkeit ist nicht unbedingt relevant, solang es nicht zu schnell ist 
\item Zukunft: Mehr K\"orperteile und mehr Kinects, Stimmerkennung hinzuf\"ugen und zusammen benutzen
\end{itemize}

\subsection{Gestenerkennung mit Leap Motion und Kinect}
Hier wird das nächste Verfahren erl\"autert. leap motion and kinect

\subsection{Gestenerkennung durch neuronale Netzwerke}
Hier wird die Gestenerkennung durch neuronale Netzwerke vorgestellt.

\subsection{Simultane Lokalisation und Erkennung von dynamischen Handgesten}
Hier wird das nächste Verfahren erl\"autert. dynamische Handgesten
\begin{itemize}
\item d
\end{itemize}

\section{Fazit}
Verfahren xy eignet sich in dieser Situation am meisten.

\begin{thebibliography}{[MT1]}
\bibitem[Kin]{Kinect}
Gu, Y., Do, H., Ou, Y., Sheng, W.:
Human Gesture Recognition through a Kinect Sensor
2014 IEEE-RAS International Conference on Humanoid Robots (2014)
% http://ieeexplore.ieee.org/stamp/stamp.jsp?arnumber=6491161
\bibitem[Nea]{Neuronal}
Barros P, Parisi G., Jirak D., Wermter S.:
Real-time Gesture Recognition Using a Humanoid Robot with a Deep Neural Architecture 2014 IEEE-RAS International Conference on Humanoid Robots (2014)
% http://ieeexplore.ieee.org/stamp/stamp.jsp?arnumber=7041431
\bibitem[Pep0]{Pepper}
Lafaye J.,Gouaillier D., Wieber P.:
Linear model predictive control of the locomotion of Pepper, a humanoid robot with omnidirectional wheels 2014 IEEE-RAS International Conference on Humanoid Robots% http://ieeexplore.ieee.org/stamp/stamp.jsp?arnumber=7041381
\bibitem[Pep1]{Pepper2}
Tanaka, F.,Isshiki K., Takahashi F.,Uekusa M., Sei R., Hayashi K.:
Pepper learns together with children: Development of an educational application 2015 IEEE-RAS 15th International Conference on Humanoid Robots (Humanoids)
% http://ieeexplore.ieee.org/document/7363546/#
\bibitem[SimLR]{LocAndRec}
Alon, J., Athitsos, V., Yuan, Q., Sclaroff, S.:
Simultaneous Localization and Recognition of Dynamic Hand Gestures Computer Sience Department 2005 IEEE Workshop on Motion and Video Computing
% http://www.cs.bu.edu/techreports/pdf/2004-012-dstw.pdf
\bibitem[Leap]{LeapMotion}
Marin, G., Dominio, F., Zanuttigh, P.:
Hand gesture recognition with leap motion and kinect devices 2014 IEEE International Conference on Image Processing (ICIP)
% http://ieeexplore.ieee.org/stamp/stamp.jsp?arnumber=7025313
\bibitem[KOMSM]{KOMSM}
Peris, M., Fukui, K.:
Both-hand gesture recognition based on KOMSM with volume subspaces for robot teleoperation 2012 IEEE International Conference on Cyber Technology in Automation, Control, and Intelligent Systems (CYBER)
% http://ieeexplore.ieee.org/stamp/stamp.jsp?arnumber=6392552
\end{thebibliography}
\end{document}
