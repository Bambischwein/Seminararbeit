% This is LLNCS.DOC the documentation file of
% the LaTeX2e class from Springer-Verlag
% for Lecture Notes in Computer Science, version 2.4
%%\documentclass{llncs}

\documentclass{llncs}
\usepackage{llncsdoc}
\usepackage[english]{babel}

\begin{document}
\markboth{Gestenerkennung in ROS mit Pepper
}{Gestenerkennung in ROS mit Pepper}
\thispagestyle{empty}
\begin{flushleft}
\LARGE\bfseries ISE Master Seminararbeit\\
in Robotik\\[2cm]
\end{flushleft}
\rule{\textwidth}{1pt}
\vspace{2pt}
\begin{flushright}
\Huge
\begin{tabular}{@{}l}
Gestenerkennung\\
in ROS\\
mit Pepper\\[6pt]
\end{tabular}
\end{flushright}
\rule{\textwidth}{1pt}
\vfill
%
\newpage
\tableofcontents
\newpage
%

\begin{abstract}
Die Gestenerkennung ist ein wichtiges Teilgebiet der Mensch-Maschine-Interaktion. Sie soll den Umgang und die Kommunikation von Menschen mit Maschinen verbessern und kommt bereits im Alltag zur Anwendung, beispielsweise bei der Nutzung von Mobiltelefonen oder Spielkonsolen. Im Rahmen dieser Arbeit werden verschiedene Verfahren zur Gestenerkennung vorgestellt und evaluiert. Darüber hinaus wird untersucht, wie gut diese Verfahren auf dem humanoiden Roboter \glqq Pepper\grqq\hspace{1pt} umzusetzen sind.

%Es werden zun\"achst drei Verfahren zur Gestenerkennung vorgestellt. Anschließen%d wird evaluiert, wie gut 



%% %%Abstract 10-15zeilen: mmi gesten als wichtige modalität gewinnbringend nutzen
%%  In Hinblick auf die Nutzung von Mensch-Maschine-Interaktion ist die Gestenerkennung eine wichtige Modalit\"at. Sie vereinfacht den Umgang und die Kommunikation von den Menschen mit einer Maschine.  
\end{abstract}
%
\section{Introduction}
In der heutigen Zeit, wo Arbeitspl\"atze, die unter anderem auch die interaktion mit Menschen beinhalten, durch humanoide Roboter ersetzt werden, gewinnt die Mensch-Maschine-Interaktion mehr an Bedeutung. Die Verfahren m\"ussen sicher und fehlerfrei funktionieren, um den Betrieb am laufen zu halten. Damit die Bedienung der Roboter vereinfacht wird, kann Gestensteuerung gewinnbringend genutzt werden.
%
\section{Gestensteuerung}
Hier kommt die Begründung hin, warum gestensteuerung benutzt wird
Anhand von Gesten k\"onnen Roboter gesteuert werden. Dadurch kann die interaktion im Alltag erleichtert werden. 
% warum gestensteuerung
%% warum gestenerkennung 
\section{Pepper}
 was kann pepper, was braucht man (eingabedatenkan\"ale ) hautfarbe, 2d, 3d andere sensoren?
\begin{itemize}
\item ca 1,20m gro\ss{} und 30kg schwer
\item vier Mikrofone, zwei RGB-Kameras und ein 3D-Sensor im Kopf
\item Gyrosensor im Torso
\item Ber\"uhrungssensoren in Kopf und H\"anden
\item Fu\ss{}bereich enth\"alt 2 Sonarsensoren, 6 Laserscanner, 3 Sto\ss{}f\"angersensoren und einen weiteren Gyrosensor.
\end{itemize}
\section{Verfahren zu Gestenerkennung}
In diesen Kapitel folgen drei Verfahren zur Gestenerkennung.
%% → verfahren, wenn mehere dann evaluieren, was passt zu szenario
%%	zielszenario: nur formen, oder bewegte gesten \"uber zeit.

\subsection{Gestenerkennung durch KOMSM}
Hier wird das KOMSM Verfahren vorgestellt.
\subsection{Gestenerkennung mit Tiefensensor}
Hier wird das Gestenerkennung mit Tiefensensor Verfahren vorgestellt.
\subsection{Gestenerkennung durch neuronale Netzwerke}
Hier wird die Gestenerkennung durch neuronale Netzwerke vorgestellt.
\section{Fazit}
Verfahren xy eignet sich in dieser Situation am meisten.
\newpage
\begin{thebibliography}{[MT1]}
\bibitem[Kin]{Kinect}
Gu, Y., Do, H., Ou, Y., Sheng, W.:
Human Gesture Recognition through a Kinect Sensor
2014 IEEE-RAS International Conference on Humanoid Robots (2014)
\bibitem[Nea]{Neuronal}
Barros P, Parisi G., Jirak D., Wermter S.:
Real-time Gesture Recognition Using a Humanoid Robot with a Deep Neural Architecture 2014 IEEE-RAS International Conference on Humanoid Robots (2014)
\bibitem[Pep0]{Pepper}
Lafaye J.,Gouaillier D., Wieber P.:
Linear model predictive control of the locomotion of Pepper, a humanoid robot with omnidirectional wheels 2014 IEEE-RAS International Conference on Humanoid Robots
\bibitem[Pep1]{Pepper2}
Tanaka, F.,Isshiki K., Takahashi F.,Uekusa M., Sei R., Hayashi K.:
Pepper learns together with children: Development of an educational application 2015 IEEE-RAS 15th International Conference on Humanoid Robots (Humanoids)
%http://ieeexplore.ieee.org/document/7363546/#
\end{thebibliography}

\end{document}
