\begin{abstract}
Die Gestenerkennung ist ein wichtiges Teilgebiet der Mensch-Maschine-Interaktion. Sie soll den Umgang und die Kommunikation von Menschen mit Maschinen verbessern und kommt bereits im Alltag zur Anwendung, beispielsweise bei der Nutzung von Mobiltelefonen oder Spielkonsolen. Im Rahmen dieser Arbeit werden verschiedene Verfahren zur Gestenerkennung vorgestellt und evaluiert. Dar\"uber hinaus wird untersucht, wie gut diese Verfahren auf dem humanoiden Roboter \glqq Pepper\grqq\hspace{1pt} umzusetzen sind.

%% %%Abstract 10-15zeilen: mmi gesten als wichtige modalität gewinnbringend nutzen
%%  In Hinblick auf die Nutzung von Mensch-Maschine-Interaktion ist die Gestenerkennung eine wichtige Modalit\"at. Sie vereinfacht den Umgang und die Kommunikation von den Menschen mit einer Maschine.  
\end{abstract}
